\documentclass[10pt, onecolumn]{jarticle}   	% use "amsart" instead of "article" for AMSLaTeX format
\usepackage{geometry}                		% See geometry.pdf to learn the layout options. There are lots.
\geometry{a4paper}                   		% ... or a4paper or a5paper or ... 
%\geometry{landscape}                		% Activate for for rotated page geometry
%\usepackage[parfill]{parskip}    		% Activate to begin paragraphs with an empty line rather than an indent
%\usepackage{graphicx}				% Use pdf, png, jpg, or eps§ with pdflatex; use eps in DVI mode
								% TeX will automatically convert eps --> pdf in pdflatex		

\usepackage[all]{xy}
\usepackage{enumerate}
\usepackage{version}
\usepackage{amssymb}
\usepackage{amsmath}
\usepackage{cases}

\usepackage[dvipdfmx]{graphicx}

\usepackage{amssymb}
\usepackage{amsthm}

\include{"texdefinition"}

\title{varsの変換}
\author{\myname}
\date{20190601}					% Activate to display a given date or no date

\begin{document}
\maketitle

\section{概要}
Expressionにλbindingを表すvarsをペアとして考えている。

clauseなので入れ子はない。

varsのサイズ=次元=変数の数

代入をtermsで表現している。


$V_1,V_2,V_3,V_4$がすべて変数のベクトルの場合、次の代入を考える。

$$V_1 = (V_2.V_3)・V_4$$

$(V_2.V_3)$は、$V_4$を$V_1$に変換するものと考えられる。

次の関係は自明。
$$|V_1| = |V_3|、|V_2| = |V_4|$$

例。
$$[a,b] = ([x,y,z].[X,Y])・ [a,b,c])$$

N次元の代入をM次元の代入に変換している。

これを使うと、clauseの変数リストをliteralの変数リストとの変換は、これを使えばできる。


ただし、literalの変数リストといっても、clauseの中のインスタンス化されたものではなく、
標準リテラルを考える必要がある。

\subsection{順序}
$V_2.V_3$と$V_3.V_2$は、次元を逆に変換する。

置換かと思ったが、置換をもっと分解した操作のような気がする。

ともかく非可換な操作。



%\subsection{定義}
%\include{"definition"}



\end{document}
