\documentclass[10pt, oneside]{jarticle}   	% use "amsart" instead of "article" for AMSLaTeX format
\usepackage{geometry}                		% See geometry.pdf to learn the layout options. There are lots.
\geometry{a4paper}                   		% ... or a4paper or a5paper or ... 
%\geometry{landscape}                		% Activate for for rotated page geometry
%\usepackage[parfill]{parskip}    		% Activate to begin paragraphs with an empty line rather than an indent
%\usepackage{graphicx}				% Use pdf, png, jpg, or eps§ with pdflatex; use eps in DVI mode
								% TeX will automatically convert eps --> pdf in pdflatex		

\usepackage[all]{xy}
\usepackage{enumerate}
\usepackage{version}
\usepackage{amssymb}
\usepackage{amsmath}
\usepackage{cases}

\usepackage[dvipdfmx]{graphicx}

\usepackage{amssymb}
\usepackage{amsthm}

\include{"texdefinition"}

\title{公理集合の変換について}
\author{\myname}
\date{20190512}							% Activate to display a given date or no date


\begin{document}
\maketitle

\section{概要}
公理集合 $\mathcal{A}$に対して、構文的変換$\mathcal{T}$を定義し、$\mathcal{T}(\mathcal{A})$が公理集合となる。

この構文的変換を公理の変換と呼ぶ。

$\mathcal{T}$は論理式ではなく、構文的な変換なので、$\mathcal{A}$と$\mathcal{T}(\mathcal{A})$に$\mathcal{T}$を加えても証明はできない。そもそも、$\mathcal{T}$は論理式ではない。

\subsection{$\mathcal{A}$と$\mathcal{T}(\mathcal{A})$の関係は?}
たとえば・・・

$$\mathcal{A}\vdash \psi ならば \mathcal{T}(\mathcal{A}) \vdash \mathcal{T}(\psi)$$
$$\mathcal{T}(\mathcal{A}) \vdash \mathcal{T}(\psi) ならば \mathcal{A}\vdash \psi $$

という意味で等価。となるようにしたい。

$\mathcal{T}$はどのようなものか? なんとなくNatural Transformationにみえるのだけど、そのためには、どのようなカテゴリーになるべきか。

%\subsection{定義}
%\include{"definition"}


\section{例}
例が必要


\end{document}
