\documentclass[10pt, onecolumn]{jarticle}   	% use "amsart" instead of "article" for AMSLaTeX format
\usepackage{geometry}                		% See geometry.pdf to learn the layout options. There are lots.
\geometry{a4paper}                   		% ... or a4paper or a5paper or ... 
%\geometry{landscape}                		% Activate for for rotated page geometry
%\usepackage[parfill]{parskip}    		% Activate to begin paragraphs with an empty line rather than an indent
%\usepackage{graphicx}				% Use pdf, png, jpg, or eps§ with pdflatex; use eps in DVI mode
								% TeX will automatically convert eps --> pdf in pdflatex		

\usepackage[all]{xy}
\usepackage{enumerate}
\usepackage{version}
\usepackage{amssymb}
\usepackage{amsmath}
\usepackage{cases}

\usepackage[dvipdfmx]{graphicx}

\usepackage{amssymb}
\usepackage{amsthm}

\include{"texdefinition"}

\title{一時的な真実}
\author{\myname}
\date{20190525}					% Activate to display a given date or no date

\begin{document}
\maketitle

\section{概要}

これまでは、FactをNNなどの帰納的なシステムで作られると想定していたが、人間がその替わりになるような例を思いついた。

Web Viewからの入力をFactとして扱うことができる。

たとえば、順番にentry画面がでてくるUIを考えると、それぞれの画面にLiteralまたはclause(タブで遷移)を考えれば、ViewにFactが対応するだろうと想像できる。

\section{もう少し具体的に}
goal指向のproverを考える。

goal($g$)の一つのリテラル($L_i$)に着目してresolutionを考える時、
その述語記号に対応するView($V_{L_i}$)を作成する。

$V$でデータを入力し、confirmしたとき、$L_i$にその入力データを反映したリテラルを
Fact($F_i$)と考える。

Factは代入($sigma_o$)と等価になる。

goal($g$)はFact($F_i$)とresolution/unify可能であり、$sigma_o$がその対価になる。
これによってresolutionに相当する操作ができる。

Viewからは、confirmだけでなくcancelでも戻りうる。その場合はresolutionに失敗したとして
そのliteralには何もしない。

\subsection{ほかの方法}
\begin{description}
\item[ clauseあるいはcoreを$(Goal, Base, Fact)$と考える] 
Baseはinput clauseで、goalはrefutationをするための手がかり、Factは普遍なものもありうるが、その時々で変わるものでもよい。上のEntryシートの例では、一年経てば年齢がかわつてしまう。そのセッションの間、変わらないという前提もありうるが、そのclause集合の表す世界が、何も変わっていないという保証はないので、Factは外部からその時々に与えられるclauseだと考えていい。
\item[ 標準literalとViewテンプレート] goalからview, viewからfactに変換するとき、標準literalが必要である。
これはgraphの$\bar{P} = P$に対応するリテラルと考えて良い。

Viewは、Logicとは独立なので、テンプレートとして持つ必要があるだろう。
これらにはvarsが付属する。

core以外にも、標準literalとViewテンプレートを用意しなくてはならない。

\item[ Fact とGraph] Factが世界についての正しい言明だとすると、Factはすべて$+P(...)$という形になる。
つまり、ground unit clause。
\item[ ground ] 世界に対する個別の言明なら、groundになる。
外部の世界の話なので、なんらかの方法によって、対象の部分集合に対する言明が得られるかもしれない。
その場合は$\forall x. +P(x) \{\phi(x)\}$ということもありうる。
\item [ Fact はルールを含むか] そうなるとFactがunit clauseである理由も希薄になる。
なんらかのルールであるとか、上に書いたような条件付きのFactというものはありうるだろう。

\item[ goal ] goalの1つのliteralが解消されるとき、複数のgoalに分裂することがある。
Viewを使わない方法なら、ないような気もするが、それはViewの形しだい。

\item[ 複数literal goal ] goalは1つのclauseであり、複数のliteralを持ちうる。
goalのView対応は、上では1つのliteralを選択して1つのViewにしたが、1つのclauseをまとめて複数のViewにする方法も考えられる。

たとえば、Viewをタブで切り替える方法にすることもできる。その場合、変数bindingによって、異なるViewの同じ変数は同じ値にbindされる。

このとき、いくつかのViewが同時にconfirmされることがある。つまり、未定義のままのViewもありうるということ

\item[ 変数の値(入力側)] goalがresolventの場合を考えるとわかりやすいが、goalのliteralのいくつかの変数にはすでに定数が代入されている場合がある。

Viewの観点からは、すでに値の決まっている項目が埋まった状態でのentryシートになる。

もともとgoalの入力値があって、それが定数になっている場合は、入力値に対する値を求めるようなスキームになる

\item[変数の値] Viewに値を代入する仕組みであるとか、Viewで入力された値からFactをつくる仕組みの部分では、標準literalとの代入$\sigma$を介して値の転送を行う。

%20190526 疲れた。ここまで
\end{description}


%\subsection{定義}
%\include{"definition"}



\section{Factの形}
ここに書いた方法では、Viewからconfirmで戻るときにFactが作られる。
それは、$+P(a,b)$という形になっている。

cancelの場合にも、Factが生成できないだろうか。
たとえば、cancelの場合、$+P(a,b)$でなく、$-P(a,b)$つまり、$P(a,b)$ではないというclauseを
生成するのはどうだろうか。

これは、入力されたデータ自体は有効になるので、通常のViewと違う解釈になる。
つまり、特定の入力の組に対して、それはViewの述語が成り立たないという強い主張をすることになる。

通常のViewなら、cancelはそれに対するFactが存在しないというだけのことだ。

$-P(a,b)$がFactであるというのは、a,bの具体的な組に対してPが成り立たないという判定をLogical Systemの外側で行なっているということになる。

そういう情報が得られることもあるはずだ。




\end{document}
