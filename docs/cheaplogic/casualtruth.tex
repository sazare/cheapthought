\documentclass[10pt, onecolumn]{jarticle}   	% use "amsart" instead of "article" for AMSLaTeX format
\usepackage{geometry}                		% See geometry.pdf to learn the layout options. There are lots.
\geometry{a4paper}                   		% ... or a4paper or a5paper or ... 
%\geometry{landscape}                		% Activate for for rotated page geometry
%\usepackage[parfill]{parskip}    		% Activate to begin paragraphs with an empty line rather than an indent
%\usepackage{graphicx}				% Use pdf, png, jpg, or eps§ with pdflatex; use eps in DVI mode
								% TeX will automatically convert eps --> pdf in pdflatex		

\usepackage[all]{xy}
\usepackage{enumerate}
\usepackage{version}
\usepackage{amssymb}
\usepackage{amsmath}
\usepackage{cases}

\usepackage[dvipdfmx]{graphicx}

\usepackage{amssymb}
\usepackage{amsthm}

\include{"texdefinition"}

\title{一時的な真実}
\author{\myname}
\date{20190525}					% Activate to display a given date or no date

\begin{document}
\maketitle

\section{概要}

これまでは、FactをNNなどの帰納的なシステムで得ると考えていたが、簡単な例として、Viewからの入力をFactとして扱うことを考えてみた。

順番にentry画面がでてくるUIを考えると、それぞれの画面にLiteralまたはclause(タブで遷移)を考えれば、ViewにFactが対応するだろうと想像できる。

\subsection{関連する部分}
goal指向のproverを考える。
goal($g$)のあるリテラル($L_i$)に着目してresolutionを考える時、
その述語記号に対応するView($V_{L_i}$)を作成する。
$V$でデータを入力し、confirmしたとき、$L_i$にその入力データを反映したリテラルを
Fact($F_i$)と考える。

Factは代入($sigma_o$)と等価になる。

goal($g$)はFact($F_i$)とresolution/unify可能であり、$sigma_o$がその対価になる。
これによってresolutionに相当する操作ができる。

Viewからは、confirmだけでなくcancelでも戻りうる。その場合はresolutionに失敗したとして
そのliteralには何もしない。


%\subsection{定義}
%\include{"definition"}



\section{Factの形}
ここに書いた方法では、Viewからconfirmで戻るときにFactが作られる。
それは、$+P(a,b)$という形になっている。

cancelの場合にも、Factが生成できないだろうか。
たとえば、cancelの場合、$+P(a,b)$でなく、$-P(a,b)$つまり、$P(a,b)$ではないというclauseを
生成するのはどうだろうか。

これは、入力されたデータ自体は有効になるので、通常のViewと違う解釈になる。
つまり、特定の入力の組に対して、それはViewの述語が成り立たないという強い主張をすることになる。

通常のViewなら、cancelはそれに対するFactが存在しないというだけのことだ。

$-P(a,b)$がFactであるというのは、a,bの具体的な組に対してPが成り立たないという判定をLogical Systemの外側で行なっているということになる。

そういう情報が得られることもあるはずだ。




\end{document}
