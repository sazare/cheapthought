\documentclass[10pt, oneside]{jarticle}   	% use "amsart" instead of "article" for AMSLaTeX format
\usepackage{geometry}                		% See geometry.pdf to learn the layout options. There are lots.
\geometry{a4paper}                   		% ... or a4paper or a5paper or ... 
%\geometry{landscape}                		% Activate for for rotated page geometry
%\usepackage[parfill]{parskip}    		% Activate to begin paragraphs with an empty line rather than an indent
%\usepackage{graphicx}				% Use pdf, png, jpg, or eps§ with pdflatex; use eps in DVI mode
								% TeX will automatically convert eps --> pdf in pdflatex		

\usepackage[all]{xy}
\usepackage{enumerate}
\usepackage{version}
\usepackage{amssymb}
\usepackage{amsmath}
\usepackage{cases}

\usepackage[dvipdfmx]{graphicx}

\usepackage{amssymb}
\usepackage{amsthm}

\include{"texdefinition"}

\title{日本語にあった述語}
\author{\myname}
%\date{}							% Activate to display a given date or no date

\begin{document}
\maketitle

\section{概要}
例えば次の文を考える。
\begin{description}
\item[ 例文1] 僕は君が好き
\end{description}

これは次のように述語化できる。
\begin{description}
\item[ 述語化1.1] 好き(僕,君)
\end{description}

しかし、日本語では語順を変えることができるのて、
\begin{description}
\item[ 例文1'] 君が僕は好き
\end{description}
と修正したとき、これを述語化1.1に変換するのは面倒くさい。

日本語で語順が変えやすいのは、助詞がその文法的な意味を肩代わりしていて、
英語では語順で文法の意味を表しているということだと思う。



そこで、助詞の「は」や「が」をキーワードとしたこのような書き方をしてはどうだろうか。
\begin{description}
\item[ 述語化1.2] 好き(は=僕,が=君)
\end{description}
こうすると、例文1'は次のようになる。
\begin{description}
\item[ 述語化1.2'] 好き(が=君,は=僕)
\end{description}

このような述語に対して、unifyを、キーワード「は」「が」での一致と考えると、語順が問題にならなくなる。

キーワードを「助詞」と考えるのは、英文では文法のカテゴリーだと考えることに相当するので
\begin{description}
\item[ 例文1e] I love you.
\end{description}

は次のように変換できる。
\begin{description}
\item[ 述語化1e] Love(I, you)
\item[ 述語化1.1e] Love(S=I, O=you)
\end{description}

こうなると、述語記号が動詞であるというのも、たまたまのような気がするので次のようにもかける。

\begin{description}
\item[ 述語化1.2e] Sent(I, love, you)
\item[ 述語化1.3e] Sent(S=I, V=love, O=you)
\end{description}
ここでは、述語は文を表していると考えているが、それは、キーワードが文法の概念だからで自然ぢと思う。

ここまでくると、データベースと似てきていて、列名がここのキーワードになる。

DBの場合は、格納されたデータとQueryのマッチングを行うので、unificationとは異なるが
マッチングするときはSQLを使うのであり、キーワードの順番は問題にならなくなっている。



\subsection{論理式}
次に論理式の場合を考える。
\begin{description}
\item[ 例文2] 君が僕にりんごを渡したので、僕はりんごを持っている
\end{description}
これは「渡す」「持つ」という述語を考えると次のような論理式になる。
\begin{description}
\item[ 述語化2] 渡す(が=君, に=僕, を=りんご) ⇒ 持つ(は=僕, を=りんご)
\end{description}

これを一般化するとこうなる。
\begin{description}
\item[ 述語化2] who1, who2, obj . 渡す(が=who1, に=who2 を=obj) ⇒ 持つ(は=who2, を=obj)
\end{description}

\begin{quotation}
日本語のparseの手を抜くため、助詞を"ha", "ga", "ni"と書くとよさそう
\end{quotation}

\subsection{文法と叙述}
このように、文法だけでは、叙述はできない。


\section{実装}
実装する場合、述語の引数をDict(Juliaの)として持てばよい。

代入文もDictとする方法を先に考えたので、何もかもDictにできないのかなとは思う。

\subsection{存在しない引数}
今度は、「私はリンゴを食べる」という文からはじめる。
\begin{description}
\item[文1] 文(ha=私, wo=リンゴ, ru=食べ)
\item[文2] [X]. 文(ha=私, wo=X, ru=食べ)
\item[文3] 文(ha=私, ru=食べ)
\end{description}

文1は、「私はリンゴを食べる」という文だが、その「リンゴ」の部分を変数にすると文2になる。
文1と文2をnifyすると$\{リンゴ \to X\}$ という代入が得られる。

これを適用すると
\begin{eqnarray*}
文(ha=私, wo=リンゴ, ru=食べ) \bullet \{リンゴ \to X\} ==\\
\forall X 文(ha=私, wo=X, ru=食べ)\bullet\{リンゴ \to X\} \\
\end{eqnarray*}


では「文3」のように「私は食べる」を述語化するとすると、unificationはどうなるべきか。

「食べる」という述語には、ha-wo-ru の3つの要素が必要だと考えるならば、who=Xを補った「文2」が正しい述語化のように見える。
文1と文3のunificationで、文に書かれていない変数Xが登場してよいのだろうか。

あえて$wo=?$を補うのだとすると、$\{wo=リンゴ\}$が代入に含まれるということか。
これは$\{リンゴ \to wo\}$ではない。

代入は、文1と文3の両方に適用して、同じ文がでてくるものでなくてはならないので
\begin{eqnarray*}
文(ha=私, wo=リンゴ, ru=食べ) \bullet \{wo=リンゴ\} ==\\
文(ha=私, ru=食べ)\bullet\{wo=リンゴ\} \\
\end{eqnarray*}

存在しない引数のリストに$\{wo=リンゴ\} $を適用すると、追加されるという規則になりそう。

\section{はたしてどうか}
日本語の述語化を考えたとき、語順を考慮して変換できるかどうか。

「孤独な文法」で考えたように、それぞれが独自のルールを使うようなシステムでは
parseの段階で、日本語文法を仮定したくなかったので、このようなことを考え始めた。

助詞に注目するのは、助詞が頻繁に出現するので、それをparseのキーワードとすることに
大きな違和感はなかったので、こちらのほうがよいと思えた。

すなくとも、parseの段階で日本語文法を考慮し、その意味に基づいて引数順序を決めるのは好まない。

この方式では、語順ではなく書かれた順番で述語の引数を書いていけば
日本語の特殊なルールを考慮しなくてもよくなる。

英文も同じようにキーワード付きにできるが、その場合、文法を考慮して順番を決めるだけなので
あまりメリットはないと思う。

また、DBのSQL文は、このような考えを持っていると思うが、先に書いたようにunificationがないので
ここの話とは様相が違うだろう。

DBは、FactとQueryに別れていて、論理式はQueryにしか書けない。

DBの中にQueryを置くのはStored Programだろうか。

Prologではないので、プログラムに相当する論理式はあるが、論理式はプログラムではない。


\section{おまけ}
キーワードを左側に書く必要もないし、述語を左側に書く必要もないのでこうもかける。

\begin{description}
\item[お1] (私=は, リンゴ=を)食べる
\item[お2] (私=は,君=が)好き
\item[お3] (大き=い、象=の, 鼻=の,話=です)
\end{description}
お3は、なにかおかしい。これは形容詞の場合で、形容詞は修飾する対象があるので、構文木のような構造が必要になりそう。
その場合は、論理式になるのだろうか。

\begin{description}
\item[お3'] $\forall X,Y . 大きい(X) \land 象(X)  \land 鼻(Y) \land の(X, Y)\land 主文(Y=の,話=です)$
\end{description}

混在している・・・

日本語の述語って何???

構文木を論理式に変換する方法は?

文法と表現の構造は別のもの

これはよくない
\end{document}


