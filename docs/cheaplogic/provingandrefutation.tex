\documentclass[10pt, onecolumn]{jarticle}   	% use "amsart" instead of "article" for AMSLaTeX format
\usepackage{geometry}                		% See geometry.pdf to learn the layout options. There are lots.
\geometry{a4paper}                   		% ... or a4paper or a5paper or ... 
%\geometry{landscape}                		% Activate for for rotated page geometry
%\usepackage[parfill]{parskip}    		% Activate to begin paragraphs with an empty line rather than an indent
%\usepackage{graphicx}				% Use pdf, png, jpg, or eps§ with pdflatex; use eps in DVI mode
								% TeX will automatically convert eps --> pdf in pdflatex		
\usepackage[all]{xy}
\usepackage{enumerate}

\usepackage[all]{xy}
\usepackage{enumerate}
\usepackage{version}
\usepackage{amssymb}
\usepackage{amsmath}
\usepackage{cases}

\usepackage[dvipdfmx]{graphicx}

\usepackage{amssymb}
\usepackage{amsthm}

\include{"texdefinition"}

\title{証明と反証、全称と存在}
\author{\myname}
\date{20190512}							% Activate to display a given date or no date

\begin{document}
\maketitle

\section{概要}
反証で証明を作る時、$\exists$と$\forall$はどのように証明プロセスに影響するか。

前にも書いたような気がするが、再整理。

Kowalskiの本をみていると、$\forall も \exists$もないclauseで直接、式を書いているので
こういう問題は気にしていないようだ。

Horn clauseに限定しているというのも理由にはありそう。

\subsection{定義}
%\include{"definition"}


\section{単純な例}
\subsection{いくつかの変数に依存して値の決まる場合}
例1)
\begin{description}
\item[ conj :] $\forall x \exists y \Phi(x,y)$
\item[ $\neg$conj :]  $\exists x \forall y \neg \Phi(x,y)$
\end{description}

conjの場合は、$\herbrand$の任意の要素$e$について、ある値yが存在して$\Phi(e,y)$が成り立つ。
これを反証で証明する場合、$\neg$conjの形になる。
clauseに変換すると、変数$x$は定数$c_x$に置き換わるので、
$$\neg \Phi(c_x, y)$$
で反証を行うことになる。

conjの証明のためには、$c_x$の選択方法にしたがって、
すべての$c_x \in \herbrand$について、反証を行う必要がある。

$\herbrand$を記号的にいくつかの集合に分け、それを表す論理式を用いて、
有限の場合分けにできれば、証明操作は完了できるだろう。

\begin{eqnarray*}
 \{\forall c_x \in \mathcal{H}\} \mathcal{A},\forall y \neg \Phi(c_x) \vdash \Box\\
 を変形し\\
 \mathcal{A} \cup \{\neg \Phi(c_x, y)\} \vdash \Box \\
 ここで\\
 \mathcal{A} \rightarrow \Phi(x, f(x))と \neg\Phi(c_x, y)をrefuteし\\
 \subst{(x,y)}{(c_x, f(c_x))} という形のfの存在を示すか、証明する必要がある。
\end{eqnarray*}

\subsection{特定の変数に対して、すべての値で真になる場合}
例2)
\begin{description}
\item[ conj :] $\exists x \forall y \Phi(x,y)$
\item[ $\neg$conj :]  $\forall x \exists y \neg \Phi(x,y)$
\end{description}

conjは、
ある$c_x \in \herbrand$が存在して、すべての$y$について$\Phi(c_x, y)$が成立すると主張している。

このタイプの証明では、$\herbrand$の全要素について反証を試みて、
ひとつも反例がないことを示さなくてはならないので、反証法は適さない。
たとえば、この例を反証するには、どのような$c_x$を選んでも、
すべての$y$について$\neg \Phi(c_x, y)$であるというのだから、
x,yの両方に$\herbrand$の要素をあてはめて反証ができないことを言うしかない。

反証の中に$\exists y$があるということは、$\herbrand$の全要素について調べなくてはならないということ。


\section{どのような論理式なら反証法で証明できるのか}

まず、$\{something\}$は、意味論に基づくメタな操作を示すことにする。
例えば、$\forall c \in \mathcal{H}$で、エルブラン宇宙のすべての要素について何かを行うとかを示す。

\subsection{その1}
$$
\xymatrix{
Conjecture & Neg Conjecture \\
\forall x P(x) \ar[d] \ar[r]& \exists x\neg{P(x)}\ar[d] \\
P(x) \ar[d]& \{\forall c \in \mathcal{H}\} \neg{P(c)}, \mathcal{A} \vdash \Box \\
\neg{P(x)} 
}
$$

\begin{description}
\item[Conjecureの場合] $\mathcal(A), \neg P(x) \vdash \Box$が証明できれば
これが反証になり、$\Box\subst{x}{c}$というmguが得られた時
$c \in H$で$\mathcal{A}\vdash P(c)$しか言えない。

すべての$H$の要素については何も証明できていない。

\item[Neg Conjectureの場合]反証から
$$\{\forall c \in \mathcal{H}\} \mathcal{A} \vdash P(c)$$
が言えるので、$\forall x P(x)$が証明できる。

\end{description}

\subsection{その2}

Conjectureが$\exists x P(x)$の場合、
\begin{description}
\item[Conjecture] の中で$\exists x P(x)$をskolemizeすると
$\{ select c \in \mathcal{H}\}$として要素cを選び、それの否定を
$\neg P(c)$としてneg conjを作り、
$\mathcal{A}, \neg P(c) \vdash \Box$を
すべてのcについて実行する必要があり、手続きとして成り立たない。
\item[Negate]では$\forall x \neg P(x)$を反証するので
skolemizeは発生せず、
$\mathcal{A}, \neg P(x)\vdash \Box$の証明をして
$\substset{x}{a}$が得られたとすると、$x$として$a$の
存在がいえるので、Conjectureの証明ができる。
このときは$\mathcal{H}$を用いた、意味論にもとづく操作は
必要ない。
\end{description}

その1とその2から、Conjecture $\Phi$をそのまま否定するのは、
手続きとして成り立たない。




\end{document}
