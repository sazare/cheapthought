\documentclass[10pt, oneside]{jarticle}   	% use "amsart" instead of "article" for AMSLaTeX format
\usepackage{geometry}                		% See geometry.pdf to learn the layout options. There are lots.
\geometry{a4paper}                   		% ... or a4paper or a5paper or ... 
%\geometry{landscape}                		% Activate for for rotated page geometry
%\usepackage[parfill]{parskip}    		% Activate to begin paragraphs with an empty line rather than an indent
%\usepackage{graphicx}				% Use pdf, png, jpg, or eps§ with pdflatex; use eps in DVI mode
								% TeX will automatically convert eps --> pdf in pdflatex		

\usepackage[all]{xy}
\usepackage{enumerate}
\usepackage{version}
\usepackage{amssymb}
\usepackage{amsmath}
\usepackage{cases}

\usepackage[dvipdfmx]{graphicx}

\usepackage{amssymb}
\usepackage{amsthm}

\include{"texdefinition"}

\title{証明と反証、全称と存在}
\author{\myname}
%\date{}							% Activate to display a given date or no date

\begin{document}
\maketitle

\section{概要}
反証で証明を作る時、$\exists$と$\forall$はどのように証明プロセスに影響するか。

前にも書いたような気がするが、再整理。

\subsection{定義}
%\include{"definition"}


\section{単純な例}
\subsection{いくつかの変数に依存して値の決まる場合}
例1)
\begin{description}
\item[ conj :] $\forall x \exists y \Phi(x,y)$
\item[ $\neg$conj :]  $\exists x \forall y \neg \Phi(x,y)$
\end{description}

conjの場合は、$\herbrand$の任意の要素$e$について、ある値yが存在して$\Phi(e,y)$が成り立つ。
これを反証で証明する場合、$\neg$conjの形になる。
clauseに変換すると、変数$x$は定数$c_x$に置き換わるので、
$$\neg \Phi(c_x, y)$$
の反証を行うことになる。

conjの証明のためには、$c_x$の選択方法にしたがって、
すべての$c_x \in \herbrand$について、反証を行う必要がある。

$\herbrand$を記号的にいくつかの集合に分け、それを表す論理式を用いて、
有限の場合分けにできれば、証明操作は完了できるだろう。

\subsection{特定の変数に対して、すべての値で真になる場合}
例2)
\begin{description}
\item[ conj :] $\exists x \forall y \Phi(x,y)$
\item[ $\neg$conj :]  $\forall x \exists y \neg \Phi(x,y)$
\end{description}

conjは、
ある$c_x \in \herbrand$が存在して、すべての$y$について$\Phi(c_x, y)$が成立すると主張している。

このタイプの証明では、$\herbrand$の全要素について反証を試みて、
ひとつも反例がないことを示さなくてはならないので、反証法は適さない。
たとえば、この例を反証するには、どのような$c_x$を選んでも、
すべての$y$について$\neg \Phi(c_x, y)$であるというのだから、
x,yの両方に$\herbrand$の要素をあてはめて反証ができないことを言うしかない。

反証の中に$\exists y$があるということは、$\herbrand$の全要素について調べなくてはならないということ。


\section{どのような論理式なら反証法で証明できるのか}

\end{document}
