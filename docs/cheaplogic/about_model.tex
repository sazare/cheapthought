\documentclass[10pt, onecolumn]{jarticle}   	% use "amsart" instead of "article" for AMSLaTeX format
\usepackage{geometry}                		% See geometry.pdf to learn the layout options. There are lots.
\geometry{a4paper}                   		% ... or a4paper or a5paper or ... 
%\geometry{landscape}                		% Activate for for rotated page geometry
%\usepackage[parfill]{parskip}    		% Activate to begin paragraphs with an empty line rather than an indent
%\usepackage{graphicx}				% Use pdf, png, jpg, or eps§ with pdflatex; use eps in DVI mode
								% TeX will automatically convert eps --> pdf in pdflatex		

\usepackage[all]{xy}
\usepackage{enumerate}
\usepackage{version}
\usepackage{amssymb}
\usepackage{amsmath}
\usepackage{cases}

\usepackage[dvipdfmx]{graphicx}

\usepackage{amssymb}
\usepackage{amsthm}

\include{"texdefinition"}

\title{モデルについて}
\author{\myname}
\date{}					% Activate to display a given date or no date

\begin{document}
\maketitle

\section{概要}
証明が1つできると、それを構成するmguは、入力clausesの中のどれかのliteral のpair である。

すべての述語は、このmguによってinstance化され、それが述語($P$)に対応するモデル($\mathsf{P}$)を決める。

各$\Contra$の証明(反証)木にぶらさがるリテラルは、それのペアから作成されるmguによって、すべて入力リテラルのインスタンスになる。

複数のリテラルからなるclauseは、単独の述語のモデルではなく、複数の述語のモデルをまとめ、加えて制約条件を不可されたものになる。

そのモデルに属する2つの述語のモデルがどのような関係にあるのかは、制約条件によって決まるが、変数を含まず、ground termのみの構成物であり、制約条件はそのモデル全体を制約するだけで、同等の他の制約条件もありうる。

cluaseの集合+goalがどのようなモデルを規定しているのかが、このメモにおける関心事である。

clauseはliteralをorで繋ぐので、個々のリテラルのモデルの和集合+制約条件になるだろう。
述語の引数の数や、clauseに出現するliteralの数によって形がかわる。

clauseの集合のモデルは、個々のclauseのモデルの共通部分になるかもしれない。
clauseのモデルに様々な形があるので、共通部分といっても何になるのかは難しい。

単純な集合と冪集合の和の集合がclauseのモデルだとすると、
そういうものの共通部分は、全リテラルのモデルの共通部分でよいようにも思え
だとするとclauseの和集合の関係が謎になる。

さらに考える必要がある


述語は複数の引数を持つものがあるので、モデルは$\{\mathsf{P} \times D^n\}$のような形になると考えられて、これに制約条件$\phi$をつけて
$$\{(P, t_1,t_2,\dots) : P \in \mathsf{P}, t_i \in H, \phi(\bar{H})\}$$
となるだろう。

本当かな??


%\subsection{定義}
%\include{"definition"}

\section{今後の課題}
\subsection{より妥当なモデルの定義}
ここで考えているモデルは不十分であり、本当にモデルとして使えるのかどうかわからない。

もっと考える必要がある。

\end{document}
