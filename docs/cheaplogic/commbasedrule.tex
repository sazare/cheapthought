\documentclass[10pt, oneside]{jarticle}   	% use "amsart" instead of "article" for AMSLaTeX format
\usepackage{geometry}                		% See geometry.pdf to learn the layout options. There are lots.
\geometry{a4paper}                   		% ... or a4paper or a5paper or ... 
%\geometry{landscape}                		% Activate for for rotated page geometry
%\usepackage[parfill]{parskip}    		% Activate to begin paragraphs with an empty line rather than an indent
%\usepackage{graphicx}				% Use pdf, png, jpg, or eps§ with pdflatex; use eps in DVI mode
								% TeX will automatically convert eps --> pdf in pdflatex		

\usepackage[all]{xy}
\usepackage{enumerate}
\usepackage{version}
\usepackage{amssymb}
\usepackage{amsmath}
\usepackage{cases}

\usepackage[dvipdfmx]{graphicx}

\usepackage{amssymb}
\usepackage{amsthm}

\include{"texdefinition"}

\title{孤独な文法}
\author{\myname}
%\date{}							% Activate to display a given date or no date

\begin{document}
\maketitle

\section{概要}
人間の言語コミュニケーションにおいて、たとえば、子供がコミュニケーションをまなんでいく過程や、まったく知らない言葉を話す環境で、人がどうその言語を学習していくのかを考えると、文法というものが先にあるわけではないことは明らか。

ではどのようにコミュニケーションが成長していくのだろうか。

「文法」というものは、気持ちの上では存在しそうに思うけれど、もしもあったとしても、「文法」をすべて完全に記述することなど、ありえない夢だ。

その文法に従うか従わないかの境界にあるような文章もまた、人は類推して理解できて、やがてその文章も「文法」に含まれる文例になるかもしれない。

もちろん、文法には例外というものもたくさんある。

だから、多くの人々に承認されたような「文法」というものを前提とせず、
もっと不完全なルールに従って、コミュニケーションが成り立つのではないだろうか。

一方で、そもそも、人間の行動は、何かのルールに従っているように見える。
あるいは、ひとりの人間は、自分のルールあるいは原則にしたがって行動する。

その「ルール」は必ずしも論理的に整合していないかもしれない。
狂信的な人々の言動は論理的であるかのように見えても、そうではない。
そこまで過激でなくても、人はしらずしらずのうちに迷信を信じているものだし
それが、重要でなければ、信じるでも信じないでもない状態で受け入れていることもあるだろう。

また、ある種の人々は、「〜だから〜」という、三段論法のようなスキーマで、他人を説得しようとし、
仲間内ではそれに成功することもある。
実際にはそのようなルールには根拠がなく、推論のように見えるものも推論とはなりえないことがある。
そのような行動をとること自体は、成長の過程でそのように教育されたからかもしれない。
学校での教育だけではなく、育てた人の行動パターンによって刷り込まれたものという意味も含めての教育。

そもそも、「規則」というものは、「ある表現」から「別の表現」へのジャンプであり、そこには何の保証もない。
人類の一部が何千年もかけて、ある種の「規則」は妥当だと考えられるという結論にたどり着いたのが「論理」であるのにすぎない。

それも「数理論理学」での論理性と、他の哲学的集団における「論理性」とが同じなのかどうかは、私は知らない。

普遍的な論理性というものは、抽象的で、生活で使われる言語表現を抽象化し普遍化したときに得られる言語表現の構造の上で成り立つ真偽についての話だと思う。

例えば、数学は抽象的な概念を扱うので、「論理学」を適用しやすい。

日常言語表現に、そのような論理的真偽を持ち込んでも役に立たないということもあるかもしれない。

論理的な推論規則が、日常の言語活動で守られないこともあるけれども
そもそも、推論の基本になるアトミックな表現が、その推論規則を適用できるように構成されていないのかもしれない。

さらに、そのような世界で、論理規則を適用することに意味があるのかどうかということも・・・。

とにかく、絶対的な文法ではなく、個人的なルールに基づくコミュニケーションの可能性について
面白そうな考えを思いついたので書いておく。
\subsection{問題点}
ルールで言われたことを理解し、ルールを拡張していく成長があるとして、
私は何をしたいのか。

たとえば、質問されたら、自分のモデルに適用して、答を得て、その答を回答する。

あるいは、質問されたら、自分のモデルでそれが成り立つかどうか、yes/noで回答する。

質問ではなく、叙述の場合は、自分のモデルを変更していく。
もし既存の部分と矛盾していれば、その叙述の確からしさを評価して
必要なら自分のモデルを変更するし、確かでないと判断すれば変更しない。

そのとき、相手にその違いを指摘して、相手のモデルを修正しようとするかどうかは
別の問題。

相手のモデルが自分のモデルと一致していないのかどうかは分からない。
メッセージが自分のモデルと両立しないということが言えるだけ。
それが言えるようなモデルシステムが必要。

\section{ルール}
人が何かを話すとき、そこには「個人的な規則/ルール」がある。

人間はそれが論理的であるかどうかにかかわらず、「ルール」を持っている。

ある人にとっては非論理的なのかもしれないが、その人にとっては「ルール」がある。

だとすれば、「ルール」は論理的あるいは整合的であることすら意味しないのだろう。

ルールが矛盾している場合、人によってはそれが耐えられず、論理的である側にルールを変更するかもしれないが、その人の信じる「論理」自体が正しいとは限らないので、人はやはり非合理の可能性をはらむ「ルール」に従う。

なぜ「ルール」に従うのか。あるいは、なぜ「ルール」を持ち出すのだろうか。
どうも「〜だから〜」というパターンのことをルールと呼ぶだけであり、それの論理的根拠などは気にしないように思う。それを「ルール」と呼ぶのか「理由」と呼ぶのかは好みの問題。

ルールというパターンに従うと、相手が反論できなくなるという成功体験があるからかもしれない。

つまり、日常生活においては、論理的ぽい説得パターンこそが人間の思考を形作る原理であり、正しさなどは二義的なものだ。

そのような思考形態がもとにあるという前提で考えると、ある言語表現を生み出すのは、基本的にこの個人的なルール(private grammer)に基づいていると思う。

といっても、そのルールの全体像は明文化されている訳ではないし、個人的なルールをすべて明文化しているわけでもない。

\subsection{Aの言葉をBが理解する過程を想像する}

Aのメッセージ$m_a$をBが自分のルールによって解読する。
そのとき、自分のどのルールにも合わなければ、$m_a$は理解不能である。

「ルールに合う」ということの意味が明確ではないが、ルール$r_b$が$m_a$に完全には適用できず、一部分にだけ適用できたとき、それでもかまわなくて、Bがそれを$i_b$として解釈したとすると、もはや適用したルールは$r_{b_1}$と変わっているが、部分的に理解したことになる。
そのとき、$m_a$の解釈できなかった部分は無視してしまう。
その場の誰も、そんなことは気にしない。

この$i_b$に基づいてBが自分の知識と照合し、自分のモデルをupdateしていくのだろう。

何かの発話については、ここにはない別の「行動目的」に沿って行うだろう。
その「行動原理」に基づいて何か伝えたいこと$w_b$があったら、Bはそれをなんらかのルール$R_b$によってメッセージ$m_b$を合成し、Aに伝える。

Aもまた同様の過程を行い、Aのもつルールから$m_b$に適用できる$r_{a_2}$を探して・・・ということになる。

このように、使っているメッセージ$m_{a,b}$の全体を司るルール/文法$G_{a,b}$というものがなくても、AとBのコミュニケーションがなりたっていくはずである。


\section{実装}
このような考えにいくらかでも妥当性があるのかどうか、シミュレーションを行ってみたい。

必要なこと
\begin{description}
\item[ メッセージ] 文字列かS式でよさそう
\item[ ルール]メッセージを解釈するルールと、
\item[ 解釈]解釈して生まれるもの。(動詞と名詞で同じ言葉を使っている)
\item[ 知識モデル] 解釈の集合が知識モデルだが、そこには構造があったりAPIがあったりする。
知識は、推論できる仕組みが必要であり、矛盾の判定ができるといい。
新しい仮説を立てることも必要。
\item[ 行動原理]ある知識のもとで、自分がどのような発話をするかは、知識のモデルやコミュニケーションのモデルとは別の動機のシステムによるのだろう。

以前考えたように、新生児は生きるための根源的な教師システムを持ち、自分の成長にともなって自己学習を続けていく。
その過程で目的がより表層的になっていくのだと思う。
そのようなシステムが必要だろうか。
\end{description}

\section{メモ}
\subsection{ルールは無数にある}
ルールは解釈のしくみ。

$\psi(T) \to (X,Y)$のようにルールはテキスト$T$を適当な変数$X$や$Y$に分解する。
これができたら「ルールを適用できた」ということになる。

$X$と$Y$があっても、$X$だけ、$Y$だけみつかってもOK。
部分的な解釈は許容され、それもまた適用できたと考える。
そのときモデルはどうなるのだろうか。

見つからなかった部分は「ノイズ」として評価してもいい。
ノイズは、コミュニケーションがうまくいっていない程度を示すだけ。
変数にとりだせたかどうかはノイズの結果ではない。
ノイズという概念は結果であり、原因とは考えない。

\subsection{ルールの原則}
ルールを選ぶという「行動原理」があると思う。
それも学習していけばよいが、どのような観点があるのかを考えてみる。
\begin{description}
\item[ (1)] 整合性(consistency)。 整合性がないのは、矛盾と呼ばれる。
既存の知識が正しいかどうかは決まっていない。
評価方法が必要。
\item[ (2)]

\end{description}

\subsection{文法とルールの違い}
文法は、多くの人が同意できるようなひとつの解釈のルールだと思う。

ここで言っているルールは、個人的なものであり、他人と共有するわけではない。
共有しなくてもコミュニケーションが可能ではないかというのが、ここで書いていることの意味。

この「個人的なルール」のことを「孤独な文法」と呼んでいるわけだ。

たまたま、他の誰かと一致しているようにみえるルールがあるかもしれない。
でもそれが同じ意味を持ったルールかどうかは分からない。
そのような「ルール」を「共有している」とは考えない。
偶然に一致しているだけであり、その一致を誰も知り得ない。

\subsection{ルールにはできるだけ適応してみる}
ルールが100個あったら、全部適用してみる。
ある程度できたらやめればいい。

\subsubsection{新しいルールをどうやって作っていくか}
この仕組みで、コミュニケーションの成長を説明したい。

\subsection{適用できた結果をモデルにまとめる}
ルールと変数と値の組を集めたもののようなものをモデルと考える。

★ Logicはとても抽象的なので、具体的な世界との間にはいろいろな仕組みが必要。

\subsection{世界の種類}
\begin{description}
\item[ (a)] 人のメッセージ
\item[ (b)] 世界についての人のメッセージ
\item[ (c)] 世界(物理的実験結果)
\end{description}


\subsubsection{異なる経緯が同じことを意味しているとは?}
Parryでも、同義語は解釈していた。
botを作るとき、同義語は基本となるようだ。
というのも、メッセージ表現の同一性を判定するとき、まずでてくるのが「同義語」だから。

とはいえ、同義語というものはそんなに簡単ではないと思うし、表現が同一かどうかは「同義語」以外にも考えなくてはならない要素がある。

ここではすこしそれを考える。

\begin{description}
\item[(A)]香林坊から有松を通って泉ヶ丘にいった」
\item[(B)]「香林坊から泉三丁目を通って泉ヶ丘にいった」
\end{description}
これらは経緯は異なるが、
「泉ヶ丘にいる」
という結果は同じになる。

(A)のあとにいる場所を尋ねる問と、(B)のあとにいる場所を尋ねる問は、同じ答になる。

あるいは、
\begin{description}
\item[ (C)] 聖徳太子は寿命で死んだ
\item[ (D)] 聖徳太子は暗殺された
\item[ (E)] 聖徳太子は崖から落ちて死んだ
\end{description}

空間的な移動でなく、できごととその結果の関係でも同じで、(D)から(E)のいずれかの知識
または前提のもとで
聖徳太子が生きているか死んでいるかという問いには「聖徳太子は死んでいる」という
回答が得られなくてはならない。

\begin{description}
\item[ 同義語] 「死んでいる」、「亡くなった」、「生きていない」などが、対話の文脈では同じ意味であることは何か?
\item[ 同義語] 「聖徳太子」が「厩戸皇子」とある文脈では同じ人物を示すとはどういうことか
\end{description}

同じであることの判断基準がある。

モデルとそれに関する表現、表明とは同じレベルの語彙が使われるとは限らない。

\subsubsection{同じ意味であること}
文を読むとき、文節点は単語になるだろう。

\begin{description}
\item[文法的語 ] 知識に依存しない
\item[意味を知っている語] 知識に依存する。既知の語。
\item[意味を知らないが、語だと分かる語] 見たことがある程度の語で、意味は分からない。 
\item[語なのかどうか判別できない何か] そのような語の存在自体がわからない
\end{description}

文を解釈するとき、既知の語に基づいて文節していく。



\subsection{モデルは問いに反応する}
なんのためにモデルを作るのかというと、
世界は問に答えてはくれないから、
世界の代わりに答をくれるような仕組みを作ろうということか。

だから、「世界とモデル」の関係と「モデルと問い」の関係の二つが問題になる。

モデルは世界を反映していなくてはならない。
つまり、ある問い(世界に対する実験でもいい)に対して、世界が与えてくれる反応が
問いをモデルの言葉に翻訳して、モデルに問い合わせると、モデルが与えてくれる反応と、
先の翻訳と同じルールによって、同じことにならないと困る。

そもそも、世界とモデルの間の翻訳が可能なのかどうかは分からないのだから
こういったことはすべて幻想かもしれない。

モデルは世界を写し取るということのほかに、問に対する答を返してくれることができないとならない。
そうでなければ、モデルが世界を反映しているかどうか、確かめることさえできない。

\subsection{ことばを作るにはモデル以外の意図が必要}
モデルをもとに、なんらかの言語表現を作るためには、
「問う者」のモデルが必要になる。

世界のモデルだけでは、誰かが何かをしたいという意志は形作られない。

それはつまり、神様のモデルでしかない。

何かを知りたいとか、世界を変化させたいとかいう、人間の欲望がなければ、言語表現は生まれない。

以前、新生児の成長を分析したように、欲望は、根源的な存在目的があって
それを達成するための仕組みがあり
そこには個体の成長を含めた世界と自分(という世界)の学習が必要になる。

現在のAIのように、その欲望自体は人間がプログラムするという場合、
欲望の整合性がどこまで達成できるのか、わからない。

それも人間が作り出せる者だろうか。

いずれにせよ、欲望があって、モデルがあって、表現したいという目的が生まれ
言語表現が登場するはずである。


人間の問いに対して回答するシステムを考えると、知識をなんらかの形で蓄積したとして、
人間の問を表面的に操作して知識から回答を得ることはできるだろうか。

表面的にというのは、文法あるいは言語表現の構造に従うだけというような意味。





\end{document}
