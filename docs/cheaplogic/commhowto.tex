\documentclass[10pt, oneside]{jarticle}   	% use "amsart" instead of "article" for AMSLaTeX format
\usepackage{geometry}                		% See geometry.pdf to learn the layout options. There are lots.
\geometry{a4paper}                   		% ... or a4paper or a5paper or ... 
%\geometry{landscape}                		% Activate for for rotated page geometry
%\usepackage[parfill]{parskip}    		% Activate to begin paragraphs with an empty line rather than an indent
%\usepackage{graphicx}				% Use pdf, png, jpg, or eps§ with pdflatex; use eps in DVI mode
								% TeX will automatically convert eps --> pdf in pdflatex		

\usepackage[all]{xy}
\usepackage{enumerate}
\usepackage{version}
\usepackage{amssymb}
\usepackage{amsmath}
\usepackage{cases}

\usepackage[dvipdfmx]{graphicx}

\usepackage{amssymb}
\usepackage{amsthm}

\include{"texdefinition"}

\title{孤独な文法の実装}
\author{\myname}
%\date{}							% Activate to display a given date or no date

\begin{document}
\maketitle

\section{概要}
基本構造
$$message \to pattern match \to model fragment$$

\begin{description}
\item[メッセージ] テキスト。文字列。
\item[ ルール] パターン + マッチング関数 + 変数名のリスト
\item[ルールの適用=知識の断片] メッセージ x ルール $\to$ ルールid + 変数の値のリスト
\item[知識モデル] 知識の断片の集合(かリスト)
\end{description}

\subsection{メッセージ}
メッセージは単純な文字列とする。
人間が受け取る情報としては、一次元に限定された文字列以上のものがある。
言葉によるコミュニケーションと並行して、視覚情報や匂い味などの多数のチャネルから得られる情報が一次元メッセージとともにやってきて、関連を持つ。
他には、表情や語調、動作、話しながらたてる音や、環境などすべて。
という意味では、メッセージはもっと複雑なのだけれど、まずは一次元の文字列をメッセージとする。
これが正しい第一歩かどうかは疑わしい。

\subsection{ルール}
ルールは、普通に言われる文法ではない。
子供は文法を理解してから話し始めるわけではない。
しかし、子供は何らかの論理性に基づいて対話をはじめる。
対話を始める前の子供の思考にも、論理性あるいは合理性はある。

ルールに、解釈の操作である「マッチング関数」も含めるのは、解釈は多様だから。

パターンは、キーワードのようなものを考えている。

\subsubsection{新生児}
新生児の場合、認知の最初が何かはむずかしいが、初期に「名前」と自分を結びつけるやりとりがある。
その前に、名前によらない存在の認識、世界の分節化の過程があるのだけれど、それはあとで考えよう。

名前を呼ばれると、読んだ人のほうに注意を向けるということは、呼ばれることが快につなげられているということだろう。
親がいろいろなアクションをするとき、名前を呼ぶこととの関連性を強調するような方法で、名前を認識させていく。
ということは、その前に、準備段階がたくさんあるということだ。

その名前と自分(self)を関連づける。同じ頃に、育てる人が分節化されていて、親は呼ばれたいという欲望から、「ママ」や「パパ」という言葉とその自分以外の何者かを結びつけるように繰り返す。
その何者かは、自分に対して「快」を与えてくれる存在であるため、この結びつきは強化されていく。

ここまでで、selfと直接selfとやりとりする誰かに名前がつく。
「ママ」と「パパ」をどのように理解しているのかは外から見ても分からない。

二つの名前を異なったものに対する呼びかけだと認識することができているのか、自分以外のものにたいする二つの呼び方があるという認識なのかは、視覚や視覚情報の認識能力がどれくらい成長しているかに依存するだろう。

これらの固有名詞(といってもいい)の学習を続けて、世界を分節化し、それぞれに名前をつけていく。
名前は単独で存在するが、新生児を律する根源的な評価システムによって、それらは分類されているはず。
その評価システムの評価は2値であり、{「快」、「苦」} という集合になる。
もしかすると、どちらでもない「無」という評価もあるかもしれない。

人間の一生はこの苦と快の周辺に形作られていく。

「名前」のルールを超えて、他のルールも学習されていく。

早い時期に学ぶ、行為はこのようなものがある。(動作ではなく行為)
言語化される行為でありっ、新生児のすべての行為に名前がつくわけではない。
たぶん、親が子供に対して行う行為が、学習されるのではないか。

\begin{description}
\item[ 与えられる] 
親が子供に食べ物やおもちゃを渡すという行為を繰り返してみせるのは、よくみかける。
親が、子供の反応を見て喜んでいるだけのような気もするが、子供はそこから「何かを与えられる」という行為を学習するのではないかと思う。
これはたいてい子供を満足させるので「快」にマッピングされるようだ。
\item[とられる]
逆に、子供から親に何かを渡すという行為もやってみせることもある。それは「取る」と認識されるのか?
そこで子供が泣くことから、これは「苦」にマッピングされるのだろう。
\end{description}

新生児の行為には、「泣く」「排泄する」「笑う」などがあるが、そこには他者は介在しないので、
言語化されるのはある程度成長してからではないか。

ここに書いていることは、間違っているかもしれない。

言葉を使い始めるのが3歳くらいだったと思うので、ここに書いていることは、言語化の準備ということではないか。

\subsubsection{旅人}
新生児ではなく、知らない国を旅して、その国の言葉を覚える旅人のことを考えてみる。

残念ながら、そのような経験がないので、ドキュメンタリーやドラマでのそのような状況から類推するしかないのだが。

まず、自分を指差して名前を連呼する、ということを通して、自分の名前あるいは呼び方を伝える。

このスキーマは、地球では一般的のようで、たいてい通じるらしい。
これが通じると、お互いの名前を連呼して、その情報が伝わったことを伝える。

そして、相手を指差して、「問う」ような表情を浮かべて、相手の名前を聞く。

教えるスキーマと、問うスキーマが同じアクションなのは、理由があって「同じスキーマ」でなければ、それが何かに対する問いであることが伝わらないからである。

$$指差し(自分) + 「太郎」$$

で自分の名前が伝わった段階で

$$指差し(X) + 問う表情$$

によって指差しているもの(X)の名前を聞いていることが相手に伝わる。
という理屈だ。

この方法の欠点は、いくつかある。
\begin{description}
\item[・]「指差し」プロトコルが共通かどうかは必ずしも確実ではない。
\end{description}
しかし、旅人は、旅の経験から、名前を聞く、妥当なプロトコルとして「指差し」を学んでいる。
となりの国でそのプロトコルが有効だったとかいう経験があるかもしれない。
\begin{description}
\item[・] 指差しているものが、同じであるかどうかは確認しようがないので、違うものの名前を教えてくれるかもしれない。
\end{description}

たぶん、「指差し」に相当する「名前を聞く」フレーズも教えてもらうことができるだろう。
「これは何」という文が、「指差し」に相当することは、どうやって学習されるのか。
非言語コミュニケーションによるかもしれない。

この方法で、生活に必要な多くの「物」の名前を知ると、文法語とともにそれらの物の関係を示す言葉を尋ねることができるようになる。

文法語と呼んでいるのは、具体的な物をさす語ではなく、文章の意味を限定していく言葉である。
日本語なら「です」「は」「から」「ない」などの、語の文章での役割や意味を示す言葉。

このように語を学習していく過程で得られる「文章パターン」は、メッセージをどう分節するのかにも依存し、それはコミュニケーションの中で得られるだろう。

対象言語のメッセージを、パターンによって分解し、そのメッセージに含まれる単語や文法構造を解読して、旅人は自分の言語に翻訳して理解する。

自分の言語に対応する概念のない語があることはどうやって認識するのか??

\subsection{ルール}
どのようなルールにも、自分を識別するルールがあるはずであり、それは例えば

$$R1:"僕"$$
$$R2:"俺"$$
$$R3:"私"$$
のような形になっている。

次のメッセージを考えよう。
$$ "僕は花が好き"$$

これに3つのルールを適用すると、ひとつだけヒットして

$$R1:僕 \to :self$$

が得られる。

:selfは自分を意味する、モデルの要素である。
新生児の場合、自己認識の発生までの手順があるはずだが、
ここではすでにあるものとする。


ルールは、適用すると、パターンに含まれる変数の値をみつけだせるものもある。
$$R4:"僕はXが好き"$$
というパターンがあり、
$$ "僕は花が好き"$$
というメッセージにこのルールを適用すると
$$R4:X \to 花$$
という変数の値が得られる。

あるいは
パターンは
$$R4^{'}: "selfはXが好き"$$
のほうがよいかも。

これで
$$R1: 僕 \to :self$$
$$R4^{'}:X \to 花$$
というフラグメントが得られる。

 (以下はまた別の機会に)

\section{ルールの適用}
知識の断片は、このルールと、適用によって得られた変数の値の組みで表すことにしよう。

\section{文脈}
言葉の表すものは、文脈によって異なる。

文脈とは何か?

たとえば、ルールを適用できた場合、そのルールに含まれる語や、適用して得られた知識の断片が
関係するモデルは、時に依存して変化する。

というのは、使われる語が増減し、減少すればそれに関連する知識は、そのあとでは使われなくなるし、
増加する場合は、新しい語が関連するような知識はそれまでにないので、古い知識が使われる可能性が減る。

このような変化による、使用できる知識の変化が「文脈」を形づくると考えられるのではないか。


\section{同義語}
類義語や同義語が同じ意味であるということもまた文脈に依存する。

以下では「同義語」という言葉に「類義語」も含めることにする。


\begin{description}
\item[語の同一性] 表現が同一の文字で書かれている。これは定義に曖昧さがない。
\item[語の類似性] 表現は同一ではないが、それの表す意味が同じか似たものであること(文脈による)

経験的に同じだと知っている場合におこる。

他の人にとって違っていてもそれは重要ではない。
語の表現が異なっているということは、誰かにとっては違った意味だということであり、
違っていなければ、意味が同じであることを考える必要がないからだ。

語の類似性は経験に依存するので、他の手段によって証明することはできない。

\item[(構造を持つ)表現の類似性] 構造が異なるのに表す意味が似ている。
論理法則などの構造に基づく同値性や包含関係に基づいて、同じであることが証明できる。
それだけではない。
\item[異なる状態での対象の関係の表現の類似性] 
二つの状態で、そこに存在する対象が似ていて、それらの間の関係が似ていれば、二つの状態は似ていると考えられる。

関係の類似性に着目するときは、対象の類似性は、あまり重要ではない。
構造の同型などを議論する場合が例。


\item[操作の類似性] 操作Aと操作Bが似ているかどうかは、結果の状態の類似性に帰着できるだろう。
初期状態が似ている時、最終状態も似ていれば、操作は似ていると考えられる。

あるいは、操作の手順が同じであるという観点もありうる。
\end{description}


\section{根源的動機}
メッセージが解釈できて、知識のモデルが変化したとして、それだけでは何も意味がない。

メッセージが質問であれば、その質問に対する回答を返すというアクションが考えられる(これもどういう動機によるのかの説明が欲しい)が、
メッセージが叙述であれば、それによる自分のモデルの修正をしても、それが何かのアクションを引き起こす必要はない。

知識のモデルだけでは、行動を引き起こすことができない。
新生児の成長と学習を考えたときには、より基本的な仕組みとして、「快」と「苦」を判定する仕組みが、まず作られているのだろうと考えた。

同じように「苦楽回路」を考えても良いが、一から作るのでなければ、もうすこし成長した後の「回路」を想定していもいいかもしれない。

たとえば、
\begin{description}
\item [完全性欲求]知識が完全に近づくと「快」と感じる回路があればいいのかもしれない。
\item [矛盾回避] 矛盾を「苦」と感じる回路があればいいのかもしれない。
\item [説明欲求] 知識が複雑になると、それをメッセージ化したくなるのはなぜだろうか。また、これによって問に対する回答を作り出そうとするだろう。
\end{description}



\subsection{メッセージ生成}
質問に対する回答は、質問の前提としている知識と、今の自分の知識の間の相違あるいは、質問に欠落している情報についての叙述になるかもしれない。

知識がいろいろなルールとそれが抽出した記号のあつまりであるならば、
質問に対する回答は
知識から関連する部分を抜き出し
それらの断片のメッセージ化(シリアライズ)と列挙を必要とするだろう。

シリアライズは一通りではない。
どのように選択するのかは問題になるし、説明が相手に理解できるかどうかを評価するとなるとまた難しくなる。



これはまたひとつのテーマなので、改めて書く。



%\subsection{定義}
%\include{"definition"}


\end{document}
