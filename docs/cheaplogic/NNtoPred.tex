\documentclass[10pt, oneside]{jarticle}   	% use "amsart" instead of "article" for AMSLaTeX format
\usepackage{geometry}                		% See geometry.pdf to learn the layout options. There are lots.
\geometry{a4paper}                   		% ... or a4paper or a5paper or ... 
%\geometry{landscape}                		% Activate for for rotated page geometry
%\usepackage[parfill]{parskip}    		% Activate to begin paragraphs with an empty line rather than an indent
%\usepackage{graphicx}				% Use pdf, png, jpg, or eps§ with pdflatex; use eps in DVI mode
								% TeX will automatically convert eps --> pdf in pdflatex		

\usepackage[all]{xy}
\usepackage{enumerate}
\usepackage{version}
\usepackage{amssymb}
\usepackage{amsmath}
\usepackage{cases}

\usepackage[dvipdfmx]{graphicx}


\usepackage{amssymb}
\usepackage{amsthm}

\include{"texdefinition"}

\title{NNから述語への変換}
\author{\myname}
%\date{}							% Activate to display a given date or no date

\begin{document}
\maketitle
\tableofcontents
$\spadesuit$は疑問のあるセクション
\newpage

\section{何を書きたいか}
\subsection{意図}
Neural Network(NN)の学習したモデルから述語表現への変換をどうするか。

単純な場合について考える。

\section{NNをどうみるか}
NNと言っているが、考えているのはCNN。複雑なのはまだ理解不足。

\subsection{ことばの定義}
\begin{description}
\item[$\bar{x}$: ] データを長さ$M$のビットベクトルと考える。1ビットはtrueともみなす。たとえば、$100\times 100$の画像なら10000ビットのベクトルになる。$\bar{x}$全体の集合を$\hat{X}$と書く。$\bar{x}$のサイズはMとする。
\item[$C_i$:] 教師データとして与えるカテゴリー。全体は$\hat{C}$と書く。カテゴリーの数はKとする。
\item[$P_i$:] $C_i$に対応する述語を$P_i$と書く。全体は$\hat{P}$と書く。
\item[$D: \hat{X} \rightarrow \hat{C}$:] $D(\bar{x}) \in \hat{C}$は、$\bar{x}$が$C_i$であると判定されることと定義する。実際にはこんなに簡単ではない。
\item[$\pi: \hat{X} \times \hat{C} \rightarrow Bool$: ] $\pi(\bar{x}, C_i)$は、データ$\bar{x}$が$C_i$である確率を示す。つまり$\Sigma_{i=1,K} \pi(\bar{x},C_i) = 1.0$。
\end{description}

手書き文字を'0'から'9'までにカテゴライズする話では、たとえばある画像 $n$が$P_i$にも$P_j$にも高い確率で仕分けされてしまう場合もある。そんなときは$D(n)$は値を二つ持つので、Dは関数ではなく、ただの対応である。

述語の観点からは$D(n) \in C_i$のとき、$P_i(n)$が真であると考える。これがFactになる。\footnote{時間まで考えると、Factは変化する}

あるいは、$\tau$を閾値として、$P_i(\bar{x}) \iff \pi(\bar{x}, i) \ge \tau$という定義もありうる。

\section{フラグメント}
$x \sqsubseteq y \iff x \land y = x$と定義する。

画像の一部のようなフラグメントを取り出す1ベクトルを$\bar{f}$で示す。

$\bar{f} \sqsubseteq \bar{x}$であり、
$\bar{f}(\bar{x}) \iff \bar{f} \land \bar{x}$
とする。

$\bar{f}(\bar{b}) \sqsubseteq \bar{b}$である。

$\bar{a}$と$\bar{b}$が与えられた時、

$f(\bar{a}) = f(\bar(b))$ ならば、
$\bar{a}$と$\bar{b}$は$f$において同じ特徴を持っているといい

$\bar{a} \sim_{f} \bar{b}$

と書くことにする。

\section{$\land$と$\lor$}

あるデータ$\bar{x}$について、$P_i(\bar{x})$ と$P_j(\bar{x})$ が成り立っている場合、$P_i \land P_j$と$P_i \lor P_j$を区別できるわけではない。

たとえば、$P_i \land P_j$は、述語記号の一種であり、$\bar{X}$の部分集合への関数に名前をつけることになる。

すでに定義された$\bar{X}$の部分集合を組み合わせて新しい集合を作る操作が$\land$や$\lor$
なので、たとえば
$$\forall x (P_i(x) \supset (P_i(x) \lor P_j(x)))$$
や
$$\forall x ((P_i(x) \and P_j(x)) \supset P_i(x)) $$
がなりたつことをNNではどう学習するのか?

そもそも、出力層に$P_i\and P_j$などというカテゴリーはないので、教師あり学習ではそのような分類はできない。
NNの外側で、論理を扱う仕組みを持つということだろうか。

それがどう役に立つのか、まだ想像がつかない(20181231)


\section{中間層の概念}
ある層$i$のノードが表す何かと、それよりも出力層に近い層$j$のノードが表す何かの間の関係も考えられる。

それぞれの層のノードに述語を対応づけられれば、できるだろう。

$P^i_p(x) \supset P^j_q(x)$

のような関係。

しかし、これも難しい。出力層の場合は、カテゴリーが確定しているのでまだ考えられるが
中間層は入力も出力も変わっていくのでどう考えればいいのか。



\section{成長}
NNでは、グラフの構造はあらかじめ与えられている。

ノードのリンクは、重みを0にすることで切り離すがことができるが、
増やすことはできない。

増やすことは、脳の成長に相当すると思う。

勿論、最初から余分にノードを与えておけばシミュレーションできるかもしれない。

生物として、もともと脳は無制限に成長しないので、それでもいいのかもしれないが
サイズと機能の関係を整理しないといけないような気がする。

\section{場と考える}
あるNNのグラフを場の表現と考えたらどうなるか。



\section{カテゴリーの逆像}
NNのシステムをひとつ考えると、それは次のような対応を与えるシステムである。

$$NN : \bar{X} , \mathcal{C} \to [0,1]$$

$\bar{X}$が入力データとなるベクトル(画像ならpixcelのベクトル)で、$\mathcal{C}$はNNのグラフを定義するときに決められた出力層のノードであると同時に、得たいカテゴリの集合である。

$[0,1]$は確率を想定していて、先に書いたように確率を閾値を決めて、$\mathcal{C} \to \mathcal{P}$として述語へのマップを定義するかもしれないが、なんらかの方法でこの対応が行われたとする。

このとき

$$P^{-1} = \{\bar{x} | P(\bar{x})\}$$

として、$P$の入力データ上の逆像を考えると

\begin{equation}
\forall x.P^{-1}_i(x) \equiv \forall x.P^{-1}_i(x) \supset P(x)
\end{equation}

という式が作れるだろう。これが、$\mathcal{A}$のルールになるのではないか。

この$P^{-1}$は、NNの行列計算の逆像になるので、計算は複雑であり、論理式を証明するのは困難だと思う。

このルールの正しさはNNに依存しており、与えられた$\bar{x}$の個々についてNNでカテゴリーを計算する必要があり、必ずしもprovedとは言えないのかもしれないが、NNのモデル自体がそれを保証しているということだろう。













\end{document}


